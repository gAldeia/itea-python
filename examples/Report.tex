\documentclass{article}%
\usepackage[T1]{fontenc}%
\usepackage[utf8]{inputenc}%
\usepackage{lmodern}%
\usepackage{textcomp}%
\usepackage{lastpage}%
\usepackage{graphicx}%
%
\usepackage[paperwidth=16cm,paperheight=12cm,tmargin=1.75cm,lmargin=1cm,rmargin=1cm,bmargin=1.5cm]{geometry}%
\usepackage[T1]{fontenc}%
\usepackage[english]{babel}%
\usepackage{datetime}%
\usepackage{grffile}%
\usepackage{booktabs}%
\usepackage{amsfonts}%
\usepackage{amssymb}%
\usepackage{amsmath}%
\usepackage{amsthm}%
\usepackage{breqn}%
\usepackage{fancyhdr}%
\usepackage{indentfirst}%
\usepackage{float}%
\title{ITEA automatic report}%
\author{\textit{ITEA\_summarizer}}%
\date{\today, \currenttime}%
\pagestyle{fancy}%
%
\begin{document}%
\normalsize%
\maketitle \vfill            
            
            Automatic report created by \textit{ITEA\_summarizer} package.
            This report makes usage of several methods to automatically inspect
            and explain the final expression found in the evolutionary process
            performed by the ITEA algorithm.

            \vfill \pagebreak%

            \lhead{Pre-execution --- ITEA automatic report}
            \chead{}
            \rhead{\today, \currenttime}
            
            \lfoot{}
            \cfoot{}
            \rfoot{\thepage\ | \pageref{LastPage}}%
\section*{Descriptive statistics of the data}%
\label{sec:Descriptivestatisticsofthedata}%

                Reporting descriptive statistics for 5
                (from a total of 8) features contained on the
                training data. The features were selected based on the absolute
                final importance.%


\begin{table}[H]%
\centering%
\footnotesize%
\begin{tabular}{lrrrrr}
\toprule
{} &     AveBedrms &        MedInc &      AveOccup &      AveRooms &      Latitude \\
\midrule
count &  13828.000000 &  13828.000000 &  13828.000000 &  13828.000000 &  13828.000000 \\
mean  &      1.097533 &      3.876745 &      3.128660 &      5.436556 &     35.651238 \\
std   &      0.445688 &      1.903102 &     12.646130 &      2.449446 &      2.134064 \\
min   &      0.333333 &      0.499900 &      0.692308 &      0.888889 &     32.550000 \\
25\%   &      1.006623 &      2.568575 &      2.432189 &      4.459802 &     33.940000 \\
50\%   &      1.049552 &      3.538750 &      2.819702 &      5.232422 &     34.270000 \\
75\%   &      1.100283 &      4.756600 &      3.282093 &      6.058566 &     37.720000 \\
max   &     25.636364 &     15.000100 &   1243.333333 &    141.909091 &     41.950000 \\
\bottomrule
\end{tabular}
%
\end{table}

%
\vfill \pagebreak

%
\section*{Algorithm Hyper-parameters}%
\label{sec:AlgorithmHyper{-}parameters}%

                The following hyperparameters were used to execute the
                algorithm. If the random\_state parameter was set to an 
                integer value (or a numpy randomState instance was given), then
                it is possible to repeat the exact execution by using the same
                training data and the parameters listed below.%
{\footnotesize \begin{verbatim}    expolim : (-1, 1)
    fitness_f : None
    gens : 75
    max_terms : 5
    popsize : 75
    random_state : 42
    simplify_method : None
    verbose : 10
    tfuncs : [log, sqrt.abs, id, sin, cos, exp]\end{verbatim} } \vfill \pagebreak

%

            \lhead{Execution --- ITEA automatic report}
            \chead{}
            \rhead{\today, \currenttime}
            
            \lfoot{}
            \cfoot{}
            \rfoot{\thepage\ | \pageref{LastPage}}
        %
\section*{Evolution convergence}%
\label{sec:Evolutionconvergence}%

                The algorithm took 162.406 seconds to
                completely run. Below are the plots for the average fitness
                of the population and the best individual fitness for each
                generation.\vfill%


\begin{figure}[H]%
\centering%
\includegraphics[width=0.8\textwidth]{/tmp/pylatex-tmp.v_djzwii/fitness_convergence.pdf}%
\end{figure}

%
\vfill \pagebreak

%
\section*{Best expression}%
\label{sec:Bestexpression}%

                The best expression corresponds to the expression with
                the best fitness on the last generation before the evolution
                ends. Not necessarily it will be the simpliest or the global
                optimum expression of the evoution. The final expression is a regressor with a fitness of
                0.69172, and the number of IT terms is
                5. Below is an LaTeX representation
                of the expression:
                
                \vfill {\small \begin{dmath}ITExpr = \underbrace{\beta_0 \cdot sqrt.abs(\frac{HouseAge \cdot Longitude}{AveRooms \cdot AveBedrms \cdot Latitude})}_{\text{term 0}} + \underbrace{\beta_1 \cdot sqrt.abs(MedInc)}_{\text{term 1}} + \underbrace{\beta_2 \cdot sqrt.abs(\frac{MedInc \cdot HouseAge \cdot AveBedrms}{AveRooms \cdot AveOccup \cdot Latitude})}_{\text{term 2}} + \underbrace{\beta_3 \cdot sqrt.abs(\frac{Longitude}{MedInc \cdot AveRooms \cdot Latitude})}_{\text{term 3}} + \underbrace{\beta_4 \cdot sqrt.abs(\frac{MedInc \cdot HouseAge \cdot Population}{AveRooms \cdot Latitude})}_{\text{term 4}} + I_0\end{dmath} } \vfill \pagebreak

%
\section*{Best expression metrics}%
\label{sec:Bestexpressionmetrics}%
On the next page is reported a table
            containing the coefficients for the previous expression, as well as
            some metrics calculated for each term individually:
            
            \begin{itemize}
            \item \textbf{coef:} coefficient of each term (or coefficients,
                  if the itexpr is an instance of ITExpr_classifier);

            \item \textbf{coef stderr:} the standard error of the coefficients;

            \item \textbf{disentang.:} mean pairwise disentanglement between
                  each term when compared with the others;

            \item \textbf{M.I.:} mean continuous mutual information between
                  each term when compared with the others;

            \item \textbf{pred. var.:} variance of the predicted outcomes for
                  each term when predicting the training data.
            \end{itemize}

             \vfill \pagebreak%


\begin{table}[H]%
\centering%
\footnotesize%
\begin{tabular}{llllrrr}
\toprule
{} &    coef &       func & coef stderr &  disentang. &   M.I. &  pred. var. \\
\midrule
term 0 &  -0.317 &   sqrt.abs &       0.009 &       0.213 &  0.301 &       0.174 \\
term 1 &   2.029 &   sqrt.abs &       0.028 &       0.253 &  0.506 &       0.840 \\
term 2 &   4.258 &   sqrt.abs &       0.068 &       0.171 &  0.280 &       0.417 \\
term 3 &   3.673 &   sqrt.abs &       0.083 &       0.256 &  0.470 &       0.341 \\
term 4 &   0.008 &   sqrt.abs &       0.001 &       0.107 &  0.205 &       0.007 \\
term 5 &  -4.299 &  intercept &       0.079 &       0.000 &  0.000 &       0.000 \\
\bottomrule
\end{tabular}
%
\end{table}

%
\vfill \pagebreak

%
\section*{Partial derivatives}%
\label{sec:Partialderivatives}%
{\footnotesize\begin{dmath}\frac{\partial}{\partial MedInc} ITExpr = \underbrace{1\beta_1 \cdot sqrt.abs'(MedInc)()}_{\text{term 1}} + \underbrace{1\beta_2 \cdot sqrt.abs'(\frac{MedInc \cdot HouseAge \cdot AveBedrms}{AveRooms \cdot AveOccup \cdot Latitude})(\frac{HouseAge \cdot AveBedrms}{AveRooms \cdot AveOccup \cdot Latitude})}_{\text{term 2}} + \underbrace{-1\beta_3 \cdot sqrt.abs'(\frac{Longitude}{MedInc \cdot AveRooms \cdot Latitude})(\frac{Longitude}{MedInc^{2} \cdot AveRooms \cdot Latitude})}_{\text{term 3}} + \underbrace{1\beta_4 \cdot sqrt.abs'(\frac{MedInc \cdot HouseAge \cdot Population}{AveRooms \cdot Latitude})(\frac{HouseAge \cdot Population}{AveRooms \cdot Latitude})}_{\text{term 4}}\end{dmath}\begin{dmath}\frac{\partial}{\partial HouseAge} ITExpr = \underbrace{1\beta_0 \cdot sqrt.abs'(\frac{HouseAge \cdot Longitude}{AveRooms \cdot AveBedrms \cdot Latitude})(\frac{Longitude}{AveRooms \cdot AveBedrms \cdot Latitude})}_{\text{term 0}} + \underbrace{1\beta_2 \cdot sqrt.abs'(\frac{MedInc \cdot HouseAge \cdot AveBedrms}{AveRooms \cdot AveOccup \cdot Latitude})(\frac{MedInc \cdot AveBedrms}{AveRooms \cdot AveOccup \cdot Latitude})}_{\text{term 2}} + \underbrace{1\beta_4 \cdot sqrt.abs'(\frac{MedInc \cdot HouseAge \cdot Population}{AveRooms \cdot Latitude})(\frac{MedInc \cdot Population}{AveRooms \cdot Latitude})}_{\text{term 4}}\end{dmath}\begin{dmath}\frac{\partial}{\partial AveRooms} ITExpr = \underbrace{-1\beta_0 \cdot sqrt.abs'(\frac{HouseAge \cdot Longitude}{AveRooms \cdot AveBedrms \cdot Latitude})(\frac{HouseAge \cdot Longitude}{AveRooms^{2} \cdot AveBedrms \cdot Latitude})}_{\text{term 0}} + \underbrace{-1\beta_2 \cdot sqrt.abs'(\frac{MedInc \cdot HouseAge \cdot AveBedrms}{AveRooms \cdot AveOccup \cdot Latitude})(\frac{MedInc \cdot HouseAge \cdot AveBedrms}{AveRooms^{2} \cdot AveOccup \cdot Latitude})}_{\text{term 2}} + \underbrace{-1\beta_3 \cdot sqrt.abs'(\frac{Longitude}{MedInc \cdot AveRooms \cdot Latitude})(\frac{Longitude}{MedInc \cdot AveRooms^{2} \cdot Latitude})}_{\text{term 3}} + \underbrace{-1\beta_4 \cdot sqrt.abs'(\frac{MedInc \cdot HouseAge \cdot Population}{AveRooms \cdot Latitude})(\frac{MedInc \cdot HouseAge \cdot Population}{AveRooms^{2} \cdot Latitude})}_{\text{term 4}}\end{dmath}\begin{dmath}\frac{\partial}{\partial AveBedrms} ITExpr = \underbrace{-1\beta_0 \cdot sqrt.abs'(\frac{HouseAge \cdot Longitude}{AveRooms \cdot AveBedrms \cdot Latitude})(\frac{HouseAge \cdot Longitude}{AveRooms \cdot AveBedrms^{2} \cdot Latitude})}_{\text{term 0}} + \underbrace{1\beta_2 \cdot sqrt.abs'(\frac{MedInc \cdot HouseAge \cdot AveBedrms}{AveRooms \cdot AveOccup \cdot Latitude})(\frac{MedInc \cdot HouseAge}{AveRooms \cdot AveOccup \cdot Latitude})}_{\text{term 2}}\end{dmath}\begin{dmath}\frac{\partial}{\partial Population} ITExpr = \underbrace{1\beta_4 \cdot sqrt.abs'(\frac{MedInc \cdot HouseAge \cdot Population}{AveRooms \cdot Latitude})(\frac{MedInc \cdot HouseAge}{AveRooms \cdot Latitude})}_{\text{term 4}}\end{dmath}\begin{dmath}\frac{\partial}{\partial AveOccup} ITExpr = \underbrace{-1\beta_2 \cdot sqrt.abs'(\frac{MedInc \cdot HouseAge \cdot AveBedrms}{AveRooms \cdot AveOccup \cdot Latitude})(\frac{MedInc \cdot HouseAge \cdot AveBedrms}{AveRooms \cdot AveOccup^{2} \cdot Latitude})}_{\text{term 2}}\end{dmath}\begin{dmath}\frac{\partial}{\partial Latitude} ITExpr = \underbrace{-1\beta_0 \cdot sqrt.abs'(\frac{HouseAge \cdot Longitude}{AveRooms \cdot AveBedrms \cdot Latitude})(\frac{HouseAge \cdot Longitude}{AveRooms \cdot AveBedrms \cdot Latitude^{2}})}_{\text{term 0}} + \underbrace{-1\beta_2 \cdot sqrt.abs'(\frac{MedInc \cdot HouseAge \cdot AveBedrms}{AveRooms \cdot AveOccup \cdot Latitude})(\frac{MedInc \cdot HouseAge \cdot AveBedrms}{AveRooms \cdot AveOccup \cdot Latitude^{2}})}_{\text{term 2}} + \underbrace{-1\beta_3 \cdot sqrt.abs'(\frac{Longitude}{MedInc \cdot AveRooms \cdot Latitude})(\frac{Longitude}{MedInc \cdot AveRooms \cdot Latitude^{2}})}_{\text{term 3}} + \underbrace{-1\beta_4 \cdot sqrt.abs'(\frac{MedInc \cdot HouseAge \cdot Population}{AveRooms \cdot Latitude})(\frac{MedInc \cdot HouseAge \cdot Population}{AveRooms \cdot Latitude^{2}})}_{\text{term 4}}\end{dmath}\begin{dmath}\frac{\partial}{\partial Longitude} ITExpr = \underbrace{1\beta_0 \cdot sqrt.abs'(\frac{HouseAge \cdot Longitude}{AveRooms \cdot AveBedrms \cdot Latitude})(\frac{HouseAge}{AveRooms \cdot AveBedrms \cdot Latitude})}_{\text{term 0}} + \underbrace{1\beta_3 \cdot sqrt.abs'(\frac{Longitude}{MedInc \cdot AveRooms \cdot Latitude})(\frac{1}{MedInc \cdot AveRooms \cdot Latitude})}_{\text{term 3}}\end{dmath}} \vfill \pagebreak

%

                \lhead{post-execution --- ITEA automatic report}
                \chead{}
                \rhead{\today, \currenttime}
                
                \lfoot{}
                \cfoot{}
                \rfoot{\thepage\ | \pageref{LastPage}}
            %
\section*{Global importances with \textit{Average partial Effects}}%
\label{sec:GlobalimportanceswithtextitAveragepartialEffects}%

                Feature importances with Average Partial Effects. This method
                attributes the importance to the i-th variable by calculating
                the average of the partial derivative w.r.t. i, evaluated for
                all data in the training set.

                \vfill%


\begin{figure}[H]%
\centering%
\includegraphics[width=0.8\textwidth]{/tmp/pylatex-tmp.v_djzwii/pe.pdf}%
\end{figure}

%
\vfill \pagebreak

%

                \lhead{post-execution --- ITEA automatic report}
                \chead{}
                \rhead{\today, \currenttime}
                
                \lfoot{}
                \cfoot{}
                \rfoot{\thepage\ | \pageref{LastPage}}
            %
\section*{Global importances with \textit{Integrated Gradients}}%
\label{sec:GlobalimportanceswithtextitIntegratedGradients}%

                Feature importance using the Average Integrated Gradients
                importances. The idea is to calculate a local
                importance score for a feature $i$ by evaluating the integral of
                the models' gradients $\frac{\partial f}{\partial x_i}$ along a
                straight line between one baseline and the specific point.
            
                \vfill%


\begin{figure}[H]%
\centering%
\includegraphics[width=0.8\textwidth]{/tmp/pylatex-tmp.v_djzwii/ig.pdf}%
\end{figure}

%
\vfill \pagebreak

%

                \lhead{post-execution --- ITEA automatic report}
                \chead{}
                \rhead{\today, \currenttime}
                
                \lfoot{}
                \cfoot{}
                \rfoot{\thepage\ | \pageref{LastPage}}
            %
\section*{Global importances with \textit{Shapley Values}}%
\label{sec:GlobalimportanceswithtextitShapleyValues}%

                Feature importance with the average approximation of the 
                Shapley values. The shapley values are based on coalition game
                theory, where players contribute differently to the team. The
                Shapley value is the total contribution of the player, and
                represents the overall contribution of the player.
            
                \vfill%


\begin{figure}[H]%
\centering%
\includegraphics[width=0.8\textwidth]{/tmp/pylatex-tmp.v_djzwii/shapley.pdf}%
\end{figure}

%
\vfill \pagebreak

%
\section*{\textit{Normalized partial Effects}}%
\label{sec:textitNormalizedpartialEffects}%

                Feature importances with Normalized Partial Effects. 
                To create this plot, first, the output interval is discretized.
                Then, for each interval, the partial effect of all samples
                in the training set that results in a prediction within the
                interval are calculated. Finally, they are normalized in
                order to make the total contribution by 100\%.

                \vfill%


\begin{figure}[H]%
\centering%
\includegraphics[width=0.8\textwidth]{/tmp/pylatex-tmp.v_djzwii/normalized_partial_effects.pdf}%
\end{figure}

%
\vfill \pagebreak

%
\section*{\textit{Partial Effects at the Means}}%
\label{sec:textitPartialEffectsattheMeans}%

                Partial Effects plots created by fixing the co-variables at
                the means and evaluating the model's output when only one
                variable changes. For simplicity, at most 5 variables are
                selected to create the plot (the 5 most important variables
                considering their Average Partial Effects).

                \vfill%


\begin{figure}[H]%
\centering%
\includegraphics[width=\textwidth]{/tmp/pylatex-tmp.v_djzwii/partial_effets_at_means.pdf}%
\end{figure}

%
\vfill \pagebreak

%
\end{document}